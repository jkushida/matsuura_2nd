\documentclass[a4paper,twocolumn,10pt]{jarticle}

\usepackage[dvipdfmx]{graphicx}
\usepackage{amsmath,amssymb}
\usepackage[linesnumbered,ruled,vlined]{algorithm2e}
\usepackage{bm}

\pagestyle{empty}
%\usepackage[top=15truemm,bottom=15truemm,left=15truemm,right=15truemm]{geometry}

%縦横のマージンは25ポイント
\usepackage[top=25truemm,bottom=25truemm,left=25truemm,right=25truemm]{geometry}



\begin{document}
\twocolumn[%
\begin{center}
{\Large\bf 複数車両同時最適化問題に対する\\ $\varepsilon$制約法における$\varepsilon$レベル制御法の改良}
\par
\medskip
{\large 松浦 隆斗(1720189)
\par
\medskip
知能工学科計算知能研究室
\par
指導:高濱徹行 教授
 串田淳一 准教授
 原章 准教授
}
\end{center}
\par
\medskip
\medskip
]


%卒業論文 要旨のフォーマット

\section{はじめに}

DEは、実数値関数を対象とした進化的アルゴリズムの一種であり
,制約なし最適化手法を制約付き最適化手法へ変換する$\varepsilon$制約法\cite{ε制約法}と組み合わせることで,様々な制約付き最適化問題を解くことができる.
$\varepsilon$制約法では$\varepsilon$レベルと呼ばれる制約の緩和度合いを探索中に減少させながら個体群を進化させる.
本論文では、個体群の探索状況から$\varepsilon$レベルを制御する方法を提案する。提案手法では、従来の減少度合いを決めるパラメータ$cp$の代わりに、現世代の個体群の制約逸脱度から$\varepsilon$レベルの値を決定する。
対象とする問題は実問題ベースの制約付き最適化問題である複数車両同時最適化問題(マツダベンチマーク問題)\cite{マツダベンチマーク問題}とし,評価実験を通して提案手法の有効性を示す.



\section{提案手法}
提案手法では,$\varepsilon$レベルの減少スケジュールをあらかじめ決定せずに,個体群の探索の状況に合わせて動的に決定する.式(\ref{提案手法})に示すように
,各世代における個体群の制約逸脱度の良い個体の上位$NP×s$番目の制約逸脱度を世代$t$の$\varepsilon$レベルとする。また、$\varepsilon$レベルが$0$に収束するように、${T}_0(0<{T}_0<{T}_{max})$世代以降は常に$0$となるようにする。
\begin{eqnarray}
\varepsilon(t)=
\left\{
\begin{array}{cc}
    {\phi({\bm x}_α)} & \mbox{$0\leq t<{T}_0$} \\
    {0} & \mbox{$t\geq{T}_0$}\\
\end{array}
\right.
\label{提案手法}
\end{eqnarray}
ここで、${x}_α$は制約逸脱度が上位$NP×s$番目の個体$(α=NP×s)$とし、${T}_0$は${T}_{max}\times{r}(r\in[0,1])$世代とする。


\section{実験結果}
マツダベンチマーク問題に提案手法を適用し、最適化性能と従来手法との比較を分析する。評価回数の上限は30000回であり、ベースベクトルの選択方法にパレートbest戦略を用いた結果を示す。従来手法と提案手法の探索結果を表\ref{tbl:パレートbest}に示す。
\begin{table}[htbp]
\begin{center}
\caption{パレートbest戦略}
\label{tbl:パレートbest}
\begin{tabular}{|c|c|c|c|c|c|}
\hline
      & Average & Std & Min & Max & Median  \\ \hline
従来手法 & 2.58395	& 0.00624 & 2.57212 & 2.60532 & 2.58270\\ \hline
提案手法 & 2.58485 & 0.00389 & 2.57741 & 2.59080 & 2.58525\\ \hline
\end{tabular}
\end{center}
\end{table}
%桁数を小さくする
%それでもはみ出るなら表の文字サイズを小さくする

従来手法との比較の結果、最適値の平均において、すべての戦略で同等の精度を得ることができた。これにより、パラメータ$cp$を減らすことできたといえる。また、21試行すべてにおいて実行可能解を得ることができたことから、提案手法によるプログラム変更の影響は少なく、安定性は保たれていると考える。


\section{まとめ}
実験結果より,提案手法の性能は従来手法と同程度であり,パラメータ$cp$を用いずに$\varepsilon$レベルを制御できることがわかった.探索性能の大きな向上は実現できなかったが、アルゴリズムをシンプルにすることができ、様々な実問題に対して効率的に解探索を行うことができると考える.今後は、他の実問題に対する性能評価や,探索の段階に応じた$s$の設定方法について検討を行う.


\bibliography{btxsample}
\bibliographystyle{junsrt}

%欧文用	和文用	特徴
%plain	jplain	参考文献をアルファベット順で出力する
%unsrt	junsrt	参考文献を引用された順で出力する


\end{document}
