\documentclass[12pt,a4paper]{jreport}
\setlength{\columnsep}{3zw}
\usepackage{color}
\usepackage[dvipdfmx]{graphicx}
\usepackage{amsmath,amssymb}
\usepackage[linesnumbered,ruled,vlined]{algorithm2e}
\usepackage{bm}
\usepackage[top=10truemm,bottom=20truemm,left=15truemm,right=15truemm]{geometry}



% LaTeXについてはこちらを参考にして下さい.
% http://mikilab.doshisha.ac.jp/dia/seminar/latex/index.html

\title{複数車両同時最適化問題に対する
\\ε制約法におけるεレベル制御法の改良}
\author{松浦隆斗(1720189)}
\date{日付: /}



\begin{document}
\maketitle
\section{はじめに}
進化的アルゴリズム(Evolutionary Algorithm, EA)は、生物が集団遺伝また進化を繰り返すことで環境に対して最適な状態になる仕組みを真似て作られた、最適化問題に対して最適解を探索するためのアルゴリズムである。詳しくは、生物のメカニズムである生殖、突然変異、遺伝子組み換え、親と子の世代交代といった進化の仕組みに着想を得たアルゴリズムを用いる。[参考]この手法は、ある問題に対して変化と選択に基づく世代交代を繰り返すことで解の集団を進化させ最適解を得る[参考]。また、最適化問題は、ある関数が与えられた領域内で最小値または最大値をとるような解を求める問題のことである。

EAは、最適化したい値である目的関数値だけを求める事ができる直接探索法であり、またプログラムの実装も容易である。EAには、遺伝的アルゴリズム(Genetic Algorithm, GA)、進化的戦略(Evolution Strategy, ES)、遺伝的プログラミング(Genetic Programming, GP)などの分類が存在する。また、これらも同じように個体を生物のように捉えることで最適解の探索をしている。

本研究で使用する最適化手法は、上記の戦略ではなく差分進化(Differential Evolution, DE)を用いる。DEは、進化的アルゴリズムの一種であり、最適化問題を解く際に用いられる手法である。[参考]また、実装が容易であり、優れた探索性能があることから広く使われる手法となっている。


最適化手法を適応させ最適化する問題の一つとして人工的に作成したベンチマーク関数問題があるが、これは実問題を反映した問題であるとは限らない。そこで、環境規制による燃費の向上のための車体重量の軽量化と、顧客ニーズの多様性にともなう多品種少量生産に適応するための共通板厚部品点数の最大化の2目的の制約付き最適化問題であるマツダベンチマーク問題がマツダ株式会社から提案された。[参考文献]実世界の最適化問題の多くは、多くの制約の下で目的関数の最適値を探索する、制約付き最適化問題であるため、マツダベンチマーク問題は実問題に取り組むうえで重要な問題の一つと言える。


本研究では、制約付き最適化問題であるマツダベンチマーク問題の車体総重量の最小化を対象にDEを用いることで最適化を行う。評価回数は、進化計算シンポジウム2017で行われたマツダベンチマーク問題のコンペティションと同様の30000回とする。[参考文献]また、DEは制約なし最適化手法であるため、マツダベンチマーク問題を直接解くことが出来ない。そこで、EAの分野で制約の対処法があり、その中の一つであるεレベル比較を使用するε制約法[参考文献]を適用したεDEを用いることで制約付き最適化問題の最適化を行う。ε制約法をDEに適用することは容易であり、また探索性能も優れていることが報告されている。[参考文献]

ε制約法では、εレベルをあらかじめ決定したパラメータによってスケジューリングし制御する。この方法では、個体群とε制約法の関連性が無いため、εレベルと個体群の進化度合のそれぞれは、独立した行動をとることが多い。これにより、個体群が実行可能領域に達する前にεレベルが0になるなど探索失敗に繋がる。また、εレベルの制御で用いるパラメータcpは、組み合わせによって実行結果が大きく変わり、実験効率に大きく関わる存在である。

(問題の制約式の改善の難易度を反映されていない)

そこで本論文では、εレベルの制御を個体群の動きに依存させるような制御をし、個体群とε制約法の関連性を作り出すことで、εレベルと個体群の進化度合いの両方を考慮しながら探索を行えるようにする、また、εレベルは個体群の制約逸脱度を利用し決定するため、パラメータcpを用いずに制御できることを示す。


本論文の構成は以下の通りである。第2章では、制約付き最適化問題であるマツダベンチマーク問題について記述する。第3章では、εDEの先行研究について説明し、(第4章では、ε制約法について説明する。第5章では、ε制約法を適用したεDEについて説明し)、第6章では、提案手法であるεレベルを個体群の探索状況にもとづいてεレベルを制御する方法について説明する。第7章で、実験結果と考察をし、第8章で、本論文のまとめと今後の課題について述べる。

\section{マツダベンチマーク問題}


\section{差分進化(Differential Evolution, DE)}

\end{document}